\documentclass[12pt]{article}
\usepackage{amsmath}
\usepackage{amsfonts}
\usepackage{mathrsfs}
\usepackage{lscape}
\usepackage{listings}
\usepackage{graphicx} % Allows for importing of figures
\usepackage{color} % Allows for fonts to be colored
\usepackage{comment} % Allows for comments to be made
\usepackage{accents} % Allows for accents to be made above and below text
%\usepackage{undertilde} % Allows for under tildes to take place for vectors and tensors
\usepackage[table]{xcolor}
\usepackage{array,ragged2e}
\usepackage{hyperref}
\usepackage{framed} % Allows boxes to encase equations and such
\usepackage{subcaption} % Allows for figures to be side-by-side
\usepackage{float} % Allows for images to not float in the document
\usepackage{booktabs}
%\usepackage[margin=0.75in]{geometry}
\usepackage[final]{pdfpages}
\usepackage{enumitem}
\usepackage[section]{placeins}

%%%%%%%%%%%%%%%%%%%%%%%%%  Function used to generate vectors and tensors %%%%%%%%%
\usepackage{stackengine}
\stackMath
\newcommand\tensor[2][1]{%
	\def\useanchorwidth{T}%
	\ifnum#1>1%
	\stackunder[0pt]{\tensor[\numexpr#1-1\relax]{#2}}{\scriptscriptstyle \sim}%
	\else%
	\stackunder[1pt]{#2}{\scriptscriptstyle \sim}%
	\fi%
}
%%%%%%%%%%%%%%%%%%%

\definecolor{mygrey}{rgb}{0.97,0.98,0.99}
\definecolor{codeblue}{rgb}{.2,0,1}
\definecolor{codered}{rgb}{1,0,0}
\definecolor{codegreen}{rgb}{0.3,0.33,0.12}
\definecolor{codegray}{rgb}{0.5,0.5,0.5}
\definecolor{codepurple}{rgb}{0.55,0.0,0.55}
\definecolor{codecyan}{rgb}{0.0,.4,.4}

\lstdefinestyle{mystyle}{
	backgroundcolor=\color{mygrey},   
	commentstyle=\color{codegreen},
	keywordstyle=\color{codeblue},
	stringstyle=\color{codepurple},
	numberstyle=\tiny\color{codegray},
	basicstyle=\footnotesize,
	breakatwhitespace=false,         
	breaklines=true,                 
	captionpos=b,                    
	keepspaces=true, 
	numbers=left,                    
	numbersep=5pt,                  
	showspaces=false,                
	showstringspaces=false,
	showtabs=false,                  
	tabsize=2
}
\lstset{style=mystyle}

\lstset{language=Matlab,backgroundcolor=\color{mygrey}}
%\usepackage{lastpage}
%\usepackage{fancyhdr}
%\pagestyle{fancy}
%\lhead{\large{Nik Benko, John Callaway, Nick Dorsett, Martin Raming}} 
%\chead{\large{\textbf{ME EN 6960: Lab 1}}}
%\rhead{\today}
%\cfoot{[\thepage\ of \pageref{LastPage}]}
%\fancyheadoffset{.5cm}
\setlength{\parindent}{0cm}
\usepackage[left=.5in, right=0.50in, top=1.00in,bottom=1.00in]{geometry}
\usepackage{microtype} 
\usepackage{setspace}
\doublespace
%%%%%%%%%%%%%%%%%%%%%%%%%%%%%%%%%%%%%%%%%%%%%%%%%%%%%%%%%%%%%%%%%%%%%%%%%%


\begin{document}
\title{ Review of manuscript ``Measurement of Stress Intensity Factors Using Digital Image Correlation" By: Group F\\ \normalsize{ME EN 6960}}
\maketitle


\section*{Synopsis:} 
In this manuscript stress intensity factors (SIF) for Mode I and mixed mode conditions around a crack are investigated by use of DIC. A  thin rectangular specimen made of PMMA with a saw-cut crack is placed in a three-point bend configuration to induce flexural stresses while imaging the face of the plate for DIC. In order to induce Mode I type stress the specimen is placed symmetrically in the three-point bend fixture and asymmetrically for mixed mode stresses.  Experimental results are calculated with displacements found from DIC. Then the Mode I experimental results are compered to theoretical approximations that were found using Westegaard's equations. Mixed mode SIF were  also calculated and presented with an estimate of mode mix percentage.
\\ 
\\
\textit{\underline{Recommendation:} Accept with minor revisions}

\section*{Comments on the technical aspects of the manuscript}
The introduction shows that a thorough literature review was done prior to implementation of analysis. However, the fundamental process in  DIC functions is not presented, which would better inform the reader of its context within modern research efforts.   The authors motivations as well as the contribution goals of the manuscript are well stated. 
 \\
 \\

\section*{Comments related to non-technical aspects of the manuscript} 

The document does have a few grammatical errors, for example in line 224 ``... there is an overall uncertainty associated with the each of the measured strain values" (the word ``the" should be left out before the word ``each").  I also noticed a few inconsistencies as is found in the ``Theory of Fracture mechanics " section where the word ``mode" also appears capitalized (``Mode"). Additionally, some sentences would read better if broken up in to separate sentences, like the sentence starting with ``ImageJ ..." in line 199. 
Section 2 could benefit from some reorganization. It seems as if the there are too many sub-sections which can become confusing to the reader in regards to context of the experiment.  Furthermore, flow and clarity could be improved by separating methods and theory from procedures as well as presenting results separate from discussion.
\end{document}