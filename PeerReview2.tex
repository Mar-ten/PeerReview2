\documentclass[12pt]{article}
\usepackage{amsmath}
\usepackage{amsfonts}
\usepackage{mathrsfs}
\usepackage{lscape}
\usepackage{listings}
\usepackage{graphicx} % Allows for importing of figures
\usepackage{color} % Allows for fonts to be colored
\usepackage{comment} % Allows for comments to be made
\usepackage{accents} % Allows for accents to be made above and below text
%\usepackage{undertilde} % Allows for under tildes to take place for vectors and tensors
\usepackage[table]{xcolor}
\usepackage{array,ragged2e}
\usepackage{hyperref}
\usepackage{framed} % Allows boxes to encase equations and such
\usepackage{subcaption} % Allows for figures to be side-by-side
\usepackage{float} % Allows for images to not float in the document
\usepackage{booktabs}
%\usepackage[margin=0.75in]{geometry}
\usepackage[final]{pdfpages}
\usepackage{enumitem}
\usepackage[section]{placeins}

%%%%%%%%%%%%%%%%%%%%%%%%%  Function used to generate vectors and tensors %%%%%%%%%
\usepackage{stackengine}
\stackMath
\newcommand\tensor[2][1]{%
	\def\useanchorwidth{T}%
	\ifnum#1>1%
	\stackunder[0pt]{\tensor[\numexpr#1-1\relax]{#2}}{\scriptscriptstyle \sim}%
	\else%
	\stackunder[1pt]{#2}{\scriptscriptstyle \sim}%
	\fi%
}
%%%%%%%%%%%%%%%%%%%

\definecolor{mygrey}{rgb}{0.97,0.98,0.99}
\definecolor{codeblue}{rgb}{.2,0,1}
\definecolor{codered}{rgb}{1,0,0}
\definecolor{codegreen}{rgb}{0.3,0.33,0.12}
\definecolor{codegray}{rgb}{0.5,0.5,0.5}
\definecolor{codepurple}{rgb}{0.55,0.0,0.55}
\definecolor{codecyan}{rgb}{0.0,.4,.4}

\lstdefinestyle{mystyle}{
	backgroundcolor=\color{mygrey},   
	commentstyle=\color{codegreen},
	keywordstyle=\color{codeblue},
	stringstyle=\color{codepurple},
	numberstyle=\tiny\color{codegray},
	basicstyle=\footnotesize,
	breakatwhitespace=false,         
	breaklines=true,                 
	captionpos=b,                    
	keepspaces=true, 
	numbers=left,                    
	numbersep=5pt,                  
	showspaces=false,                
	showstringspaces=false,
	showtabs=false,                  
	tabsize=2
}
\lstset{style=mystyle}

\lstset{language=Matlab,backgroundcolor=\color{mygrey}}
%\usepackage{lastpage}
%\usepackage{fancyhdr}
%\pagestyle{fancy}
%\lhead{\large{Nik Benko, John Callaway, Nick Dorsett, Martin Raming}} 
%\chead{\large{\textbf{ME EN 6960: Lab 1}}}
%\rhead{\today}
%\cfoot{[\thepage\ of \pageref{LastPage}]}
%\fancyheadoffset{.5cm}
\setlength{\parindent}{0cm}
\usepackage[left=.5in, right=0.50in, top=1.00in,bottom=1.00in]{geometry}
\usepackage{microtype} 
\usepackage{setspace}
\doublespace
%%%%%%%%%%%%%%%%%%%%%%%%%%%%%%%%%%%%%%%%%%%%%%%%%%%%%%%%%%%%%%%%%%%%%%%%%%


\begin{document}
\title{ Review of manuscript ``Measurement of Stress Intensity Factors Using Digital Image Correlation" By: Group F\\ \normalsize{ME EN 6960}}
\maketitle


\section*{Synopsis:} 
In this manuscript stress intensity factors (SIF) for Mode I and mixed mode conditions around a crack are investigated by use of DIC. A  thin rectangular specimen made of PMMA with a saw-cut crack is placed in a three-point bend configuration to induce flexural stresses while imaging the face of the plate for DIC. In order to induce Mode I type stresses the specimen is placed symmetrically in the three-point bend fixture and asymmetrically for mixed mode stresses.  Experimental results are calculated with displacements found from DIC. Then the Mode I experimental results are compered to theoretical approximations that were found using Westegaard's equations. Mixed mode SIF were  also calculated and presented with an estimate of mode mix percentage.
\\ 
\\
\textit{\underline{Recommendation:} Accept with minor revisions}

\section*{Comments on the technical aspects of the manuscript}
The introduction shows that a thorough literature review was done prior to implementation of analysis. However, the fundamental process by which  DIC functions is not presented in the introduction, which I belive would better inform the reader of its context within modern research efforts. The authors motivations as well as the contribution goals of the manuscript are well stated. 
 \\
 \\
 The purpose of figure 7 is somewhat unclear. Either more description or perhaps a different image would add clarity here. Figures 11-14 could be better understood with some additional discussion. 
 \\
 \\
 In section 4.2 it is stated the non-linear effects of the experimental data are due possibly to error associated with DIC. However, there was no justification for this assumption.  It could be possible that the specimen exhibited non-linear deformation as many materials in the real world do. 

\section*{Comments related to non-technical aspects of the manuscript} 

The document does have a few  typographical errors, for example in section 2 in the third paragraph the number ``1" in the sentence ``...coordinates with respect to the crack tip 1." should be omitted. Additionally, there are some inconsistencies such as,  ``Mode I" versus ``mode I".  Errors discussed with the experiment through out section 4.2 could be placed in section 4.3 to help readers associate all possible errors and uncertainties with the experiment.










\end{document}